%!TEX TS-program = pdflatex
%!TEX TS-options = -shell-escape
\newcommand{\setfontsize}{11pt}
\documentclass[%
	paper=a4,
	fontsize=\setfontsize,
	ngerman
	]{scrartcl}

% Basics für Codierung und Sprache
% ===========================================================
	\usepackage{scrtime}
	\usepackage{etex}
	\usepackage{shellesc}
	\usepackage[final]{graphicx}
	\usepackage[utf8]{inputenc}
	\usepackage{babel}
	\usepackage[german=quotes]{csquotes}
% ===========================================================

% Fonts und Typographie
% ===========================================================
	\usepackage{sourcecodepro}
	\usepackage[default]{sourcesanspro}
	\usepackage{nimbusmononarrow}
	
	\usepackage[babel=true,final,tracking=smallcaps]{microtype}
	\DisableLigatures{encoding = T1, family = tt* } % keine Ligaturen für Monospace-Fonts
	\usepackage{ellipsis}
% ===========================================================

% Farben
% ===========================================================
	\usepackage[usenames,x11names,final]{xcolor}
	\definecolor{fbblau}{HTML}{3078AB}
	\definecolor{mediumgray}{gray}{.65}
	\definecolor{blackberry}{rgb}{0.53, 0.0, 0.25}
% ===========================================================

% Mathe-Pakete und -Einstellungen
% ===========================================================
	\usepackage{mathtools}
	\usepackage{amssymb}
	\usepackage[bigdelims]{newtxmath}		% moderne Mathe-Font
	\allowdisplaybreaks						% seitenübergreifende Rechnungen
	\input{../MathCmds.tex}
	\usepackage{bm}
	\usepackage{wasysym}
% ===========================================================

% TikZ
% ===========================================================
	\usepackage{tikz}
	\usepackage{tikz-cd}					% kommutative Diagramme
	\usetikzlibrary{arrows.meta}			% mehr Pfeile!
	\usetikzlibrary{shadows}
	\usetikzlibrary{calc}
	\tikzset{>=Latex}						% Standard-Pfeilspitze
% ===========================================================

% Seitenlayout, Kopf-/Fußzeile
% ===========================================================
	\usepackage{scrpage2}
	\pagestyle{scrheadings}
	\usepackage[top=3cm, bottom=3cm, left=2.5cm, right=2cm]{geometry}
	\clearscrheadfoot 
	\setheadsepline{0.4pt} 					% Linie in Kopfzeile
	\setfootsepline{0.4pt}
	\automark[section]{section}				% Abschnittstitel in Kopfzeile
	\setkomafont{pagehead}{\bfseries}
	\setkomafont{pagefoot}{\normalfont\footnotesize}
	\cfoot{\thepage}
	\raggedbottom
	\usepackage{setspace}					
	\onehalfspacing							% Zeilenabstand 1.5-fach
	\setlength{\parindent}{0pt}
	\setlength{\parskip}{0.5\baselineskip}
	\usepackage[all]{nowidow}
% ===========================================================

% Hyperref
% ===========================================================
	\usepackage[%
		hidelinks,
		pdfpagelabels,
		bookmarksopen=true,
		bookmarksnumbered=true,
		linkcolor=black,
		urlcolor=SkyBlue2,
		plainpages=false,
		pagebackref,
		citecolor=black,
		hypertexnames=true,
		pdfauthor={Phil Steinhorst},
		pdfborderstyle={/S/U},
		linkbordercolor=SkyBlue2,
		colorlinks=false,
		backref=false]{hyperref}
	\hypersetup{final}
% ===========================================================

% Listen und Tabellen
% ===========================================================
	\usepackage{multicol}
	\usepackage[shortlabels]{enumitem}
	\setlist{itemsep=0pt}
	\setlist[enumerate]{font=\sffamily\bfseries}
	\setlist[itemize]{label=$\triangleright$}
	\usepackage{tabularx}
% ===========================================================

% Zu Testzwecken
% ===========================================================
	\usepackage{lipsum}
% ===========================================================

% Rechnerstrukturenquatsch
% ===========================================================
	\usepackage{karnaughmap}
% ===========================================================

% ntheorem
% ===========================================================
	\usepackage[amsmath]{ntheorem}
	
	\theoremstyle{default}
	\theoremseparator{.}
	\theorembodyfont{\normalfont}
	\theorempreskip{2em}
	\theorempostskip{2em}
	\newtheorem{aufg}{Aufgabe}

% minted
% ===========================================================
\usepackage{minted}
\setminted{%
	style=bw,
	fontsize=\normalsize,
	breaklines,
	breakanywhere=false,
	breakbytoken=false,
	breakbytokenanywhere=false,
	breakafter={.,},
	autogobble,
	numbersep=3mm,
	tabsize=4,
	frame=lines
}
\setmintedinline{%
	style=bw,
	fontsize=\normalsize,
	numbers=none,
	numbersep=12pt,
	tabsize=4,
	%bgcolor=gray!15,
}

\usepackage[tikz]{mdframed}
\newcommand{\code}[1]{\texttt{#1}}
\lohead{C++-Übungsaufgaben (2)}
\rohead{22.12.2018}
\rofoot{\jobname.tex}
\lofoot{}
\begin{document}

Um die Aufgaben zu bearbeiten und die eigenen Lösungen zu überprüfen, bietet es sich an, eine Funktion zu benutzen, die den Inhalt eines Arrays auf der Konsole ausgibt.
Zum Beispiel folgende:

\inputminted{cpp}{printArray.cpp}

\textbf{Aufgabe 1.} \\
Implementieren Sie eine Funktion \mintinline{cpp}{compareArray(int* a, int* b, int length)}, die zwei gleich lange Arrays \texttt{a} und \texttt{b} sowie ihre Länge \texttt{length} entgegennimmt.
Sie soll \mintinline{cpp}{true} zurückgeben, wenn beide Arrays identisch sind, andernfalls \mintinline{cpp}{false}.

\textbf{Aufgabe 2.} \\
Was leistet die folgende Funktion?
Betrachten Sie den Aufruf \mintinline{cpp}{psum(a,5)} für das Array \texttt{a = [5, 11, -2, 6, -8, 42]} und geben Sie für jede Iteration der \mintinline{cpp}{for}-Schleife den Inhalt von \mintinline{cpp}{result} an.

\inputminted{cpp}{psum.cpp}

\textbf{Aufgabe 3.} \\
Implementieren Sie eine Funktion \mintinline{cpp}{zip(int* a, int* b, int length)}, die zwei gleich lange Arrays \texttt{a} und \texttt{b} sowie ihre Länge \texttt{length} entgegennimmt.
Sie soll ein Array zurückgeben\footnote{Hiermit ist ein Pointer auf den Beginn des Arrays gemeint. Der Rückgabetyp ist also \texttt{int*}.}, das abwechselnd die Elemente von \texttt{a} und \texttt{b} enthält. \\
\textit{Beispiel:} Ist \texttt{a = [1, 2, 3, 4, 5]} und \texttt{b = [32, 33, 34, 35, 36]}, so soll die Funktion das Array \linebreak \texttt{[1, 32, 2, 33, 3, 34, 4, 35, 5, 36]} zurückliefern.

\newpage
\textbf{Aufgabe 4.}
\begin{enumerate}[a)]
	\item Implementieren Sie eine Funktion \mintinline{cpp}{count(int* a, int length, int x)}, die ein Array \texttt{a} sowie seine Länge \texttt{length} und eine weitere Zahl \texttt{x} entgegennimmt.
	Die Funktion soll zurückgeben, wie oft \texttt{x} in \texttt{a} vorkommt. \\
	\textit{Beispiel:} Ist \texttt{a = [32, 17, 1, -5, 8, 17]}, so soll \mintinline{cpp}{count(a,6,17)} das Ergebnis \texttt{2} liefern.
	\item Implementieren Sie eine Funktion \mintinline{cpp}{filter(int* a, int length, int x)}, die ein Array \texttt{a} sowie seine Länge \texttt{length} und eine weitere Zahl \texttt{x} entgegennimmt.
	Die Funktion soll ein Array zurückgeben, das alle Elemente von \texttt{a} enthält, die nicht \texttt{x} sind (in der gleichen Reihenfolge). \\
	\textit{Beispiel:} Ist \texttt{a} wie oben, so soll \mintinline{cpp}{filter(a,6,17)} das Array \texttt{[32, 1, -5, 8]} zurückgeben. \\
	\textbf{Hinweis:} Nutzen Sie die Funktion \texttt{count} aus Aufgabenteil a), um die Länge des resultierenden Arrays zu bestimmen.
\end{enumerate}
\end{document}